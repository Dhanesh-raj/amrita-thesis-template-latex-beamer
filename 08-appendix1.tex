%!TEX root = ../thesis.tex
% ******************************* Thesis Appendix A ****************************
\chapter{Program Code Python} 

\section*{Code for Wireless Network Simulation}

\subsection*{Code for Propagation Loss}
\begin{verbatim}
import heapq

def calculate_distances(graph, starting_vertex):
    distances = {vertex: float('infinity') for vertex in graph}
    distances[starting_vertex] = 0

    pq = [(0, starting_vertex)]
    while len(pq) > 0:
        current_distance, current_vertex = heapq.heappop(pq)

        # Nodes can get added to the priority queue multiple times. We only
        # process a vertex the first time we remove it from the priority queue.
        if current_distance > distances[current_vertex]:
            continue

        for neighbor, weight in graph[current_vertex].items():
            distance = current_distance + weight

            # Only consider this new path if it's better than any path we've
            # already found.
            if distance < distances[neighbor]:
                distances[neighbor] = distance
                heapq.heappush(pq, (distance, neighbor))

    return distances


example_graph = {
    'U': {'V': 2, 'W': 5, 'X': 1},
    'V': {'U': 2, 'X': 2, 'W': 3},
    'W': {'V': 3, 'U': 5, 'X': 3, 'Y': 1, 'Z': 5},
    'X': {'U': 1, 'V': 2, 'W': 3, 'Y': 1},
    'Y': {'X': 1, 'W': 1, 'Z': 1},
    'Z': {'W': 5, 'Y': 1},
}
print(calculate_distances(example_graph, 'X'))

\end{verbatim}